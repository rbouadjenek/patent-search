The patent system is designed to encourage disclosure of new technologies and novel ideas by granting exclusive rights on the use of inventions to their inventors, for a limited period of time. An important requirement for a patent to be granted is that the invention, it describes, is novel which means there is no earlier patent, publication or public communication of a similar idea. To ensure the novelty of an invention, patent offices as well as other Intellectual Property (IP) service providers perform searches called 'prior art searches' or ‘validity searches’. Since the number of patents in a company's patent portfolio affects the company market value, well-performed prior art searches are of high importance~\citep{piroi2013overview}.\\\\
\noindent
\textbf{Patent Retrieval }
\ \\
Evaluation of patent retrieval was proposed in NTCIR-2 in 2001~\citep{leong2001patent}. Since then patent retrieval has featured as a track in all NTCIR campaigns. The \emph{Clef-Ip Lab} and its tasks have evolved considerably over the last five years, from a rough approximation of a prior art search task in 2009, to, in 2013, a good simulation of the passage-level search carried out by patent searchers~\citep{piroi2013passage}. Patent retrieval is of interest in IR research since it is of commercial interest and is a challenging IR task with different characteristics to popular IR tasks such as precision-orientated ad hoc search on news archives or web document collections. Various patent search tasks have been created in mentioned campaigns including:
\\
\textbf{\textit{Ad-hoc search:}} A number of topics are used to search a patent collection with the objective of retrieving a ranked list of patents that are relevant to this topic~\citep{iwayama2003overview}.\\\\
\textbf{\textit{Invalidity search.}} The claims of a patent are considered as the topics, and the objective is to search for all relevant documents (patents and others) to find whether the claim is novel or not~\citep{joho2010survey, fujii2004overview}. All relevant documents are needed, since missing only one document can lead to later invalidation of the claim or the patent itself.
\\\\
\textbf{\textit{Passage Search:}} The same as invalidity search, but because patents are usually long,
the task focuses on indicating the important fragments in the relevant documents~\citep{fujii2007overview}.\\\\
\textbf{\textit{Prior-art Search:}}
This is the main search task carried out in patent offices; it is concerned with finding all relevant patents that are potential to invalidate the novelty of a patent application or at least that have common parts to that patent~\citep{roda2010clef}. The full patent application submitted to the patent office is considered as the topic, and patent citations that are identified by the patent office are taken as the relevant documents, therefore the objective is to find these citations of patents automatically. Prior-art search in patent retrieval focuses on finding any kind of patents relevant to the patent application in hand; this is different from invalidity search which focuses on finding any type of document that proves that a given claim in a patent application is not novel.\\\\
\textbf{Main challenges of prior-art search}
\\ \
\textit{\textbf{Query length:}} The query is a full patent application instead of just keywords which are short. So it is not focused on information need. The problem with a full document as a query is that, it might refer to multiple topics. Even in the case of a single invention, different components of the new device or process which may be described in verbose patent application. \\\\
\textit{\textbf{Recall-oriented retrieval task:}} Prior-art search is a recall-oriented retrieval task, where not missing a relevant document is more important than retrieving a small number of the most relevant documents at the top rank. Usually, in prior-art search, the patent examiners will carefully examine the first 100 or 200 documents retrieved by the search engine instead of browsing just the top few results. Missing one patent could result in a multimillion dollar lawsuit due to a patent infringement~\citep{arampatzis2007access, magdy2010pres}. \\\\
\textit{\textbf{Term Mismatch: }} The biggest challenge in patent retrieval is the significant term mismatch between the query and relevant document (Due to: Usage of new inventive words, Rewording to avoid repetition, Non-standardized acronyms: invented by authors, Synonyms: signal and wave). Magdy~\citep{magdy2010exploring} reported from their analysis presented for CLEF-IP 2009 prior-art search that 12\% of the relevant patents do not share any terms in common with patent topics after filtering out stop words ~\citep{magdy2011study}.
\begin{figure}[htpb]
   \centering
   \includegraphics[width=\textwidth,height=35mm]{figs/webprior.jpg}
   \caption{The main differences between patent prior-art search and an standard web search are: (1) the user is an expert (professional patent examiner), (2) the query is a full application not just keywords, and (3) it is a recall oriented task.}  
   \label{fig:compareappr} 
\end{figure}
\FloatBarrier 
\noindent
\textbf{Problem Statement :}
Applying standard information retrieval (IR) techniques to patent search is not effective and needs applying supplementary methods to improve its effectiveness. Although lots of methods has been proposed in recent years but still reported results for different tasks of patent search show lower retrieval effectiveness compared to other IR applications \citep{lupu2013patent}. For example, it is generally expected to achieve a mean average precision (MAP) less than 0.1, which is still regarded as an acceptable level of effectiveness. The results of various evaluation campaigns ~\citep{lupu2013patent,joho2010survey, roda2010clef, DBLP:conf/clef/PiroiLHSMF12} concluded that patent search task is certainly not a solved problem and many challenges in applying IR solutions in the intellectual property domain remain to be overcome. In this research, we focus on enhancing possible general and patent-specific IR techniques to improve the effectiveness of baseline patent prior-art search. 

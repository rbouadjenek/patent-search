We investigate the influence of term selection on retrieval
performance on the CLEF-IP Prior Art test collection, starting with
the Description section of the reference patent and using LM and BM25
scoring functions. We find that an oracular relevance feedback system
which extracts terms from the judged relevant documents far
outperforms the baseline and performs twice as well on MAP as the best
competitor in CLEF-IP 2010.  We find a very clear term selection value
threshold for use when choosing terms.  We also noticed that most of
the useful feedback terms are actually present in the original query
and hypothesized that the baseline system could be substantially
improved by removing negative query terms.
%Furthermore, a similar oracular query restricted
%to select terms from only the reference patent performs nearly as well
%as unrestricted term selection suggesting that query reduction methods
%should suffice for state-of-the-art performance on CLEF-IP 2010.
We tried four simple automated approaches to identify negative terms
for query reduction but we were unable to improve on the baseline
performance with any of them.  Finally, we show that a simple, minimal
feedback interactive approach where terms are selected from only the
first retrieved relevant document outperforms the best result from
CLEF-IP 2010.

%Patent prior-art search aims to find all relevant patents which may invalidate the novelty of a patent application or at least have common parts with patent application and should be cited. Patent search has been the centre of attention in IR communities for years, however it has lower retrieval effectiveness compared to other IR applications. In this work, we focused on the causes of failure rather than solutions. We started with relevance feedback to get a golden standard, then we concentrated on heuristics correlate with our RF standard. Finally, we showed that features other than relevance feedback can not be helpful because they are a complex mixture of useful words and noisy words. Finally, we got a considerable improvement by user feedback with a minimum effort.      
% 

We developed a baseline IR system for patent prior-art search on the top of
the Lucene open source search engine %
\footnote{\texttt{http://lucene.apache.org/}%
}, which processes queries using the probabilistic
model Okapi BM25 \cite{Robertson1993} as well as LM (Dirichlet
smoothing, and Jelinek-Mercer smoothing) \cite{Zhai2001}. %We consider 
%using BM25F~\cite{robertson2004simple} but had no opportunity to learn appropriate field weights.
We used Lucene to index the English subset of CLEF-IP 2010 dataset%
\footnote{\texttt{http://www.ifs.tuwien.ac.at/\textasciitilde{}clef-ip/}%
} with the default settings using the Porter stemming algorithm \cite{Porter1980} and English stop-word removal. 
We also removed patent-specific stop-words as described in \cite{magdy2012toward}.
CLEF-IP 2010 contains 2.6 million patent documents, and the English
test sets of CLEF-IP 2010 correspond to 1303 topics (queries). In
our implementation, each section of a patent (title, abstract, claims,
and description) is indexed in a separate field. However, when a query 
is processed, all indexed fields are targeted with an equal weight, since this generally
offers best retrieval performance. We also used the International
Patent Classification (IPC) codes assigned to the topics to filter
the search results by constraining them to have common IPC codes with
the patent topic as suggested in previous works \cite{lopez2010patatras}.
Although this IPC codes filter may prevent retrieval of relevant patents, we
have chosen to keep it for the following reasons: (i) more than 80\%
of the patent queries share an IPC code with their associated relevant
patents, and (ii) it makes the retrieval process much faster. The accuracy of the results is evaluated using two popular metrics, the Mean Average Precision (MAP) and the Average Recall, on the top 100 results for each query, assuming that patent examiners are willing to assess the top 100 patents \cite{joho2010survey}. 

We achieved the best performance while querying with the Description
section as in previous work \cite{xue2009transforming} and using
either the LM or the BM25 scoring functions. We call this initial
query: \textit{Patent Query}, and we use it as our main baseline.

In addition, we compare our results to \textit{PATATRAS}, a fully-engineered system developed by Lopez and Romary~\cite{lopez2010patatras}, which achieved the best performance in the CLEF-IP 2010 competition with a MAP=0.27. This system uses multiple retrieval models (especially Kullback-Leibler divergence~\cite{Baeza-Yates2011} and Okapi BM25) and exploites patent metadata and citation structures . The MAP we provide is comparable with \textit{PATATRAS} since we are using roughly the same settings.
%did not include errors related to non-English patents in our evaluation and we mainly focus on errors related to term matching for English patents.  
%However, the MAP we provide is not directly comparable since we excluded 22 topics from our evaluation because their relevant patents were not in English or had no IPC code matched with the topic. We pruned these topics out because we were only interested in analysing errors related to term matching. Removing 22 topics caused only 0.04 improvement in our baseline system, which is negligible. Hence, we use Top CLEF-IP 2010 results for a rough comprasion."}
% 
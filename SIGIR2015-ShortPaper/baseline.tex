We developed a baseline system for patent prior-art search using
Lucene%
\footnote{\texttt{http://lucene.apache.org/}%
}, which supports queries using the probabilistic
model Okapi BM25 \cite{Robertson1993} as well as language models (LM: Dirichlet
smoothing, and Jelinek-Mercer smoothing) \cite{Zhai2001}. We used
this system to index the English subset of CLEF-IP 2010 dataset%
\footnote{\texttt{http://www.ifs.tuwien.ac.at/\textasciitilde{}clef-ip/}%
} with the default settings for stemming and stop-word removal. 
% What do the default settings mean??? -Scott
We
also removed patent-specific stop-words as described in \cite{magdy2012toward}.
CLEF-IP 2010 contains 2.6 million patent documents, and the English
test sets of CLEF-IP 2010 correspond to 1303 topics (queries). In
our implementation, each section of a patent (title, abstract, claims,
and description) is indexed in a separate field. However, when a query
is processed, all indexed fields are targeted, since this generally
offers best retrieval performance. We also used the International
Patent Classification (IPC) codes assigned to the topics to filter
the search results by constraining them to have common IPC codes with
the patent topic as suggested in previous works \cite{lopez2010patatras}.
Although this IPC codes filter may fail to retrieve relevant patents, we
have chosen to keep it for the following reasons: (i) more than 80\%
of the reference patent queries share an IPC code with their associated relevant
patents, and (ii) it makes the retrieval process much faster. We evaluate
the results for the top 100 retrieved patents by Mean Average Precision
(MAP) and Average Recall. We assume that users examine the top 100
patents \cite{joho2010survey}.

We achieved the best performance while querying with the Description
section as in previous work \cite{xue2009transforming} and using
either the LM or the BM25 scoring functions. We call this initial
query: \emph{Patent Query}, and we use it as our main baseline.

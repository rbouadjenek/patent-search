% Reda: please have a look at this section and consider revising
%       as appropriate.  -Scott


% NOTE: It's critical to clearly and concretely delineate what we do that no one else has done. I've tried to do this in the first two sentences.

% NOTE: I've tried to make it more clear why we're mentioning each paper, i.e., it's relevance to query reformulation!

In this work, we focused on the development of an oracular query in
order to address a number of fundamental questions regarding query reformulation
and their efficacy in terms of approximating the oracular query.
Previous works have not formulated such an oracular query, but nonetheless
have inspired our investigation of query reformulation techniques 
as we briefly discuss below.
 
%Our work is different from pioneer studies on patent retrieval, as we
%looked at the patent prior art problem from the term analysis perspective to figure out the main 
%causes that generic IR methods do not work effectively in patent domain. 
%However, we cite here few works that focus mainly on query reformulation for patent search. 
%Xue and Croft \cite{xue2009transforming} conducted a series of experiments in order to examine the effect of different patent fields, and concludes with the observation that the best MAP is achieved using the text from the description section. 
%Also, 
%Fuji \cite{Fujii2007}
%showed that retrieval effectiveness can be improved by combining IR
%methods with the result of citation extraction.
Bashir et al. \cite{Bashir2010} proposed query expansion with pseudo-relevance
feedback that used machine learning for term selection.
%Terms are selected using a using a machine learning approach, by picking terms that may have a potential positive impact on the retrieval effectiveness. 
Verma and Varma
\cite{Verma2011} proposed a different approach, which instead of using
the patent text to query, use its IPC codes, which are expanded using the citation network.
Itoh et al. \cite{Itoh2003} proposed a new term selection method using different term
frequencies depending on the genre in the NTCIR-3 Patent Retrieval Task.
% The formed query is used to perform an initial search. The results are then re-ranked using queries constructed from patent text.
%Magdy et al.~\cite{magdy2011study} studied works on query expansion 
%%in patent retrieval 
%and discussed that standard query expansion techniques are
%less effective, where the initial query is the full texts of query
%patents. 
Mahdabi et al.~\cite{Mahdabi2013} used term proximity
information to identify expansion terms. Ganguly et
al.~\cite{ganguly2011patent} adapted pseudo-relevance feedback for
query reduction by decomposing a patent application into constituent
text segments and computing the Language Modelling (LM) similarities
of each segment from the top ranked documents. The least similar
segments to the pseudo-relevant documents removed from the query,
hypothesizing it can increase the precision of retrieval. Kim et
al.~\cite{kim2014diversifying} provided diverse query suggestion using
aspect identification from a patent query to increase the chance of
retrieving relevant documents. Magdy et al.~\cite{magdy2011study} 
%studied 
%works on 
%query expansion and 
discussed that standard query expansion techniques are
less effective in patent retrieval, where the initial query is the full texts of query
patents. 
%Mahdabi et al.~\cite{mahdabi2014patent}
%used linked-based structure of the citation graph together with IPC
%classification to improve the initial patent query. 
% Finally, other works investigated query suggestion for patent prior art search, which reflect real-life scenario of examiners, who form reproducible boolean queries \cite{Adams2011,Azzopardi2010,Kim2011}.












\begin{comment}
Our work is different from pioneer studies on patent retrieval, as we
closely looked into the problem rather than solutions to figure out
the causes that generic IR models which are based on term matching
process, do not work efficiently in patent domain. However, we cite here few works that focus mainly on query reformulation for patent search. 
In \cite{Itoh2003}, authors proposed a new term selection method using different term
frequencies depending on the genre in the NTCIR-3 Patent Retrieval Task.
Xue and Croft \cite{xue2009transforming} conducted a series of experiments
in order to examine the effect of different patent fields, and concludes
with the observation that the best MAP is
achieved using the text from the description section. Also, Fuji \cite{Fujii2007}
showed that retrieval effectiveness can be improved by combining IR
methods with the result of citation extraction.
Bashir et al. \cite{Bashir2010} propose query expansion with pseudo-relevance
feedback. Terms are selected using a using a machine
learning approach, by picking terms that may have a potential positive
impact on the retrieval effectiveness. Verma and Varma
\cite{Verma2011} propose a different approach, which instead of using
the patent text to query, use its IPC codes, which are expanded using the citation network. The formed
query is used to perform an initial search. The results are then re-ranked
using queries constructed from patent text.
Magdy et al.~\cite{magdy2011study} studied works on query expansion in patent
retrieval and discussed that standard query expansion techniques are
less effective, where the initial query is the full texts of query
patents. Mahdabi et al.~\cite{Mahdabi2013} used term proximity
information to identify expansion terms. Ganguly et
al.~\cite{ganguly2011patent} adapted pseudo relevance feedback for
query reduction by decomposing a patent application into constituent
text segments and computing the Language Modelling (LM) similarities
of each segment from the top ranked documents. The least similar
segments to the pseudo-relevant documents removed from the query,
hypothesizing it can increase the precision of retrieval. Kim et
al.~\cite{kim2014diversifying} provided diverse query suggestion using
aspect identification from a patent query to increase the chance of
retrieving relevant documents. Mahdabi et al.~\cite{mahdabi2014patent}
used linked-based structure of the citation graph together with IPC
classification to improve the initial patent query. 
Finally, other works investigated query suggestion for patent prior
art search, which reflect real-life scenario of examiners, who form
reproducible boolean queries \cite{Adams2011,Azzopardi2010,Kim2011}.
\end{comment}

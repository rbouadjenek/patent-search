Patents are used by legal entities to legally protect their
inventions and represent a multi-billion dollar industry of licensing
and litigation. In 2013, 302,948 patent applications were approved
in the US alone%
\footnote{http://www.uspto.gov/web/offices/ac/ido/oeip/taf/ us\_stat.htm%
}, a number that has doubled in the past 15 years. Given that a single
existing patent may invalidate a new patent application, helping inventors
assess the patentability of an idea through a patent prior art search
before writing a complete patent application is an important task.

Patent prior art search involves finding previously granted patents
that may be relevant to a new patent application. The objective and
challenges of standard formulations of patent prior art search are
different from those of standard text and web search since \cite{Magdy2012}:
(i) queries are (partial) patent applications, which consist of documents
with hundreds or thousands of words organized into several sections,
while typical queries in text and web search constitute only a few
words; and (ii) patent prior art search is a recall-oriented task,
where the primary focus is to retrieve all relevant documents at early
ranks, in contrast to text and web search that are precision-oriented,
where the primary goal is to retrieve a subset of documents that satisfy
the query intent. Another important characteristic of patent prior
art search is that, in contrast to scientific and technical writers,
patent writers tend to generalize and maximize the scope of what is
protected\textcolor{red}{{} }by a patent and potentially discourage
further innovation by third parties, which further complicates the
task of formulating effective queries. For instance, abstract and
vague terms are sometimes pre-referred to concrete ones, e.g., recording
means vs. recording apparatus; resources vs. battery life; machines
located at point of sale locations vs. vending machines, etc.

\begin{comment}
A patent is a set of exclusive rights granted to an inventor to protect their invention for a limited period of time. An important requirement for a patent to be granted is that the invention, it describes, is novel which means there is no earlier patent, publication or public communication of a similar idea. To ensure the novelty of an invention, patent offices as well as other Intellectual Property (IP) service providers mainly perform a search called `prior art search'. The purpose of `prior art search' is finding all relevant patents which may put the patent application at the risk of novelty invalidation or at least have common parts with patent application and should be cited~\cite{magdy2012toward}~\cite{piroi2013overview}. 

Patent retrieval has three main characteristics which makes it difficult compared to other IR applications: (1) the search starts with a query as long as a full patent application that helps users --usually patent examiners, inventors, or lawyers-- avoid spending long hours to formulate a query; (2) it is recall-oriented, where not missing relevant documents is more important than appearing relevant documents at top of the list; (3) unlike the web application in which authors tend to highlight their work to be easily found through search engines, authors of the patents prefer to use a vague language to avoid the invalidation of their idea.     
\end{comment}

Many works has been conducted to improve the patent retrieval effectiveness so far. However, either the results showed quite small improvement or the proposed methods were complicated and computationally expensive. Overall, the works on patent search fall in five main categories~\cite{lupu2013patent} query reformulation(query expansion and query reduction), query term selection, query suggestions, using patent meta-data and images for retrieval~\cite{lupu2013evaluating}, and Cross-Language Information Retrieval~\cite{magdy2014studying}.

%Applying standard information retrieval (IR) techniques to patent search is not effective and needs applying supplementary methods to improve the effectiveness. Although lots of methods have been proposed in recent years, reported results for different tasks of patent search show lower retrieval effectiveness compared to other IR applications~\cite{lupu2013patent}.  
In this work, we mainly emphasized on the problem from the term analysis perspective which ended in an effective minimal relevance feedback method. We investigated the influence of term selection on retrieval performance on the CLEF-IP Prior Art test collection, starting with the Description section of the reference patent and using LM and BM25 scoring functions. We found that an oracular relevance feedback system which extracts terms from the judged relevant documents far outperforms the baseline and  performs twice as well on MAP as the best competitor in CLEF-IP 2010.  We find a very clear term selection value threshold for use when choosing terms.  A much more realistic approach in which feedback terms are extracted only from the first relevant document retrieved, still outperforms the winner.   We noticed that most of the useful feedback terms are actually present in the original query and hypothesized that the baseline system could be substantially improved by removing negative query terms.  We tried three different approaches to identifying negative terms but were unable to improve on the baseline performance with any of them.

\chapter{Conclusions}
\label{cha:conc}
%\textcolor{red}{Mona: I will compose this chapter after we confirmed about all chapters!}\\
%Summary your thesis and discuss what you are going to do in the future in Section~\ref{sec:future}.

%\section{Overview}
%\label{sec:overview}
%\section{Summary}
%\label{sec:summary}

In this thesis, we investigated the reasons that patent prior art searches 
are less effective than the other web search applications.
% on CLEF-IP 2010 data collection. 
We started with recognising errors due to data curation and baseline settings that 
make small portions of the whole retrieval errors. However, the main portion of the errors are 
due to term matching process of retrieval ranking functions. 
Hence, we looked at the patent prior art search from
a term selection perspective. While previous works proposed
different solutions to improve retrieval effectiveness, we 
focused on term analysis of the patent query and top-100 retrieved patents. 
After defining an Oracular Query based on
relevance judgements, we established both the sufficiency
of the standard LM retrieval scoring models and query reduction 
methods to achieve state-of-the-art patent prior art
search performance. After finding that automated methods 
for query reduction approaches fail to offer significant
performance improvements, we showed that we can double
the MAP with minimum user interaction by approximating
the Oracular Query through a relevance feedback approach
with a single relevant document. Given that such simple 
interactive methods for query reduction with a standard LM
retrieval model outperform highly engineered patent-specific
search systems from CLEF-IP 2010, we concluded that interactive 
methods offer a promising avenue for simple but
highly effective term selection in patent prior art search.
 

\section{Contributions}
\label{sec:contributions}
We briefly summarise the major contributions of our work as follows:
\begin{enumerate}
\item \textbf{Development of an Oracular term selection system: }We built an oracular term selection system from known relevance judgements to formulate an oracular query that far outperformed the baseline and the best-performed competitor on CLEF-IP 2010. 
Experiments related to oracular system suggests the necessity of precise query
reduction and term selection techniques to improve the effectiveness of patent
prior art search.
\item \textbf{Analysis of automated query reduction techniques for patent prior art search: } We examined four simple query reduction methods to select the positive terms and prune out negative terms. We illustrated that these approaches are inefficient because they can not discriminate between useful terms and noisy terms. Since our system is over-sensitive to the existence of noisy terms, we could not achieve high performance via these simple methods. 
\item \textbf{Proposal of a semi-interactive method for query term selection: }We showed that a simple minimal interactive relevance feedback approach, where terms are selected by only the first retrieved relevant document performs as well as a highly engineered patent-specific system on CLEF-IP 2010. 
\end{enumerate}

\section{Future Work}
\label{sec:future}
In this research, we analysed the key reasons that generic IR methods are not effective for patent prior art through various experiments, which may open further research on the topic of prior art search. Thus, we describe the limitations and discuss further improvements. 
\subsection{Exploring other Term Scoring Methods}
\label{subsec:ExploringTermScoringMethods}
Our term scoring method inspired by Rocchio optimal query~\citep{manning2008introduction}. We used this score to select query terms, which resulted in a remarkable improvement in the performance. However, exploring other existing term scoring techniques like Kullback-Leibler divergence~\citep{Baeza-Yates2011} may improve the results.
\subsection{Exploring more Sophisticated Query Reduction Methods}
\label{subsec:SophisticatedQueryReduction}
One of the most important findings of our research was the existence of useful terms sufficiently inside the reference Patent Query. 
We showed that a query formulated by selecting these terms considerably outperforms the baseline. 
Thus, in this thesis, we tried four simple query reduction techniques. However, we only got slight improvement over the baseline because the retrieval models are over-sensitive to noisy terms and our proposed reduction approaches were incapable of discriminating useful terms and noisy terms. 
Given the necessity of a precise query term selection technique that can differentiate useful terms from noisy terms, 
applying more sophisticated query reduction approaches --- e.g, query term selection technique proposed in~\citep{maxwell2013compact} using affinity graph and random walk --- is an important open area of research for query term selection in patent prior art search.     

\subsection{Considering Phrasal Concepts for Query Reformulation }
\label{subsec: PhraseAnalysis}
Our research was limited to only single terms in patent documents. 
However, one important characteristic of patents is that 
inventors use longer technical terms to describe their research ideas. 
Hence, phrasal concepts and terminology 
are frequently used as keywords in target patent documents.
Hence, an obvious extension of this work is extracting phrasal concepts while reformulating the query. 

\subsection{Patent Retrieval Using Meta-data Social Information}
\label{subsec: Meta-dataNetworkAnalysis}
A retrieval based on meta-data social information and social network analysis 
is a proper alternate to a traditional IR based on term matching process, 
when the retrieval problem based on term matching is difficult --- like patent prior art search.  
Bibliographic meta-data in the XML file of a patent document contains details about 
its inventor, organisation, and other information, which can give rise in more effective retrieval. 
Recent studies aim at improving the IR process with
information coming from social networks. This is commonly known as social IR. 
Regarding this social network structure of patents, it is possible to 
find possible prior works in the social profile of other inventors with the same research interest.
Also competitive organisations may have developed the same or very close idea prior to the novel 
idea claimed in a patent application.



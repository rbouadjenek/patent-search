\chapter{Conclusions}
\label{cha:conc}
%\textcolor{red}{Mona: I will compose this chapter after we confirmed about all chapters!}\\
%Summary your thesis and discuss what you are going to do in the future in Section~\ref{sec:future}.

%\section{Overview}
%\label{sec:overview}
%\section{Summary}
%\label{sec:summary}

In this thesis, we investigated the reasons that make patent prior art searches 
less effective than other web search applications.
% on CLEF-IP 2010 data collection. 
We started with recognising errors due to data curation and baseline settings that 
make only small portion of the whole errors. We hypothesized that the main portion of the errors is 
due to term matching process of retrieval ranking functions. 
Hence, we looked at the patent prior art search from
a term selection perspective. While previous studies proposed
different solutions to improve retrieval effectiveness, we 
focused on analysing terms in the patent query and top 100 retrieved patents. 
After defining an oracular query based on
relevance judgements, we established both the sufficiency
of the standard LM retrieval scoring models and query reduction 
methods to achieve state-of-the-art patent prior art
search performance. After finding that automated methods 
for query reduction approaches fail to offer significant
performance improvements, we showed that we can double
the MAP with minimum user interaction by approximating
the oracular query through a relevance feedback approach
with a single relevant document. Given that such simple 
interactive methods for query reduction with a standard LM
retrieval model outperform highly engineered patent-specific
search systems from CLEF-IP 2010, we concluded that interactive 
methods offer a promising avenue for simple but
highly effective term selection in patent prior art search.
 

\section{Contributions}
\label{sec:contributions}
We briefly summarise the major contributions of our work as follows:
\begin{enumerate}
\item \textbf{Development of an oracular term selection system: }We built an oracular term selection system from known relevance judgements to formulate an oracular query that far outperformed the baseline and the best-performed competitor on CLEF-IP 2010. 
Experiments related to the oracular system suggested the necessity of precise query
reduction and term selection techniques to improve the effectiveness of patent
prior art search.
\item \textbf{Analysis of automated query reduction techniques for patent prior art search: } We examined four simple query reduction methods to select the positive terms and to prune the negative terms out. We illustrated that these approaches were inefficient because they could not discriminate between useful terms and noisy terms. Since our system was over-sensitive to the existence of noisy terms, we could not achieve high performance via these simple methods. 
\item \textbf{Proposal of a semi-interactive method for query term selection: }We showed that a simple minimal interactive relevance feedback approach, where terms are selected by only the first retrieved relevant document performs as well as a highly engineered patent-specific system on CLEF-IP 2010. 
\end{enumerate}

\section{Future Work}
\label{sec:future}
In this research, we analysed the key reasons making generic IR methods ineffective for patent prior art through various experiments that may open further research on the topic of prior art search. We describe the limitations and discuss further improvements as follows: 
\subsection{Exploring Other Term Scoring Methods}
\label{subsec:ExploringTermScoringMethods}
Our term scoring method inspired by Rocchio optimal query~\citep[p.181]{manning2008introduction}. We used this score to select query terms that resulted in a remarkable improvement in the performance. However, exploring other existing term scoring techniques like Kullback-Leibler divergence~\citep{Baeza-Yates2011} may improve the results.
\subsection{Exploring More Sophisticated Query Reduction Methods}
\label{subsec:SophisticatedQueryReduction}
We demonstrated that useful terms in the patent query are sufficient for an effective retrieval.
We showed that a query, formulated by a precise selection of useful terms, considerably outperforms the baseline and PATARAS. We could not approximate the oracular query by automated techniques 
because the retrieval models are over-sensitive to noisy terms and our proposed reduction approaches were incapable of discriminating between useful terms and noisy terms. 
Hence, we need more sophisticated query term selection techniques, which differentiate useful terms from noisy terms. 
For example, query term selection technique, proposed by~\cite{maxwell2013compact} using affinity graph and random walk, can be applied for patent prior art search.     

\subsection{Considering Phrasal Concepts for Query Reformulation }
\label{subsec: PhraseAnalysis}
Our research was limited to only single terms in patent documents. 
However, one important characteristic of patents is that 
inventors use longer technical terms to describe their research ideas. 
Hence, phrasal concepts and terminology 
are frequently used as keywords in target patent documents.
Hence, an obvious extension of this work is extracting phrasal concepts while reformulating the query. 

\subsection{Patent Retrieval Using Meta-data Social Information}
\label{subsec: Meta-dataNetworkAnalysis}
A retrieval based on meta-data social information and social network analysis 
is a proper alternative to a traditional IR based on term matching process 
when the retrieval problem based on term matching is difficult.
Patents are rich in meta-data; for example 
the bibliographic meta-data in the patent XML file contains details about 
its inventor, organisation, and other information that can build a multidimensional graph.
Recent studies aim at improving the IR process with
information coming from social networks; this is commonly known as social IR. 
Regarding this social network structure of patents, we can  
find possible prior works in the social profile of other inventors with the same research interest.
Also competitive organisations may have developed the same or very close idea prior to the  
idea, which is claimed in the~patent~application.



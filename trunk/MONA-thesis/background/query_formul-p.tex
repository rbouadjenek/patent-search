%Query or topic is the the request for information in any retrieval system. An effective query can lead in retrieving the required information. Queries formulated by users are not usually optimal for retrieval process, therefore, they need to be transformed by techniques such as query expansion or query reduction. 
The patent prior art search begins with a full patent application as a query. A full text as a query is a challenge compared to a classical IR, since it is not focused on the information that the user needs. In order to achieve good retrieval results, it is important to extract the best representative text with the proper weights. Therefore, query generation based on query document is essential to reduce the difficulty of formulating effective queries by users. 
%The query created before the retrieval can be modified or enhanced after retrieval. In this section, the initial query formulation will be discussed and in the next section, post-retrieval query reformulation will be covered. 
\\\\
\textbf{Which fields in patent application are more effective to extract query terms}
\ \\
A special characteristic of patent documents is their structural information. They mainly have different fields such as title, abstract, description, and claims. Different fields use different type of language for describing the invention. Abstract and description use more technical terminology while claim field usually uses a legal jargon. Structured indexing keeps the field structure in the index, which allows searching specific fields instead of searching in full document. Separate fields for meta-data (Section \ref{sec:metadata} ) like IPC code and author can help to retrieval effectiveness~\citep{magdy2010exploring}. 

Early patent search tasks mainly considered claims to build the query, the same as what examiners start the novelty process~\citep{konishi2005query, takaki2004associative, mase2005proposal, fujii2007enhancing}, whereas recent works have showed that building queries from description field is more useful in patent retrieval (considering background summary in US patents equivalent to description field in European patents.)~\citep{xue2009transforming, xue2009automatic, mahdabi2011building}. Another research showed that extracting terms according to $ log(tf)idf $ scores from every field of the query patent, and giving higher importance to terms extracted from the abstract, claims, and description fields than to terms extracted from the title field, is an effective way of constructing a search query~\citep{cetintas2012effective}. The other experiment showed that discarding description from query improves MAP up to 30\% because the description contains more noise than information~\citep{gobeill2010simple}. They also showed that claims are more informative and title is poorly informative in retrieval.   


\paragraph{Using Phrases instead of Terms}
\ \\
Most of query formulation techniques rely on terms, but encouraging results have been obtained using phrases recently~\citep{becks2010phrases}. Early results demonstrated that an NLP-based grouping of terms can increase the performance compared to the bag-of-words approach, though the increase is smaller than in a non-patent collection~\citep{osborn1997evaluating}. Another task could improve retrieval effectiveness by adding syntactic phrases in the form of dependency triples, to a bag-of-words representation~\citep{d2011combining}. Key Phrase Extraction (KPE) algorithms is another way to form a query based on phrases. A list of phrases, generated by a KPE algorithm, can succinctly represent a complex and lengthy patent. ~\citep{verma2011applying}.

\paragraph{Diverse Query Generation}
\ \\
In this approach, the focus is on generating diverse queries that can improve overall retrieval effectiveness in sessions rather than generating a single best query that can retrieve more relevant documents from a single retrieval result (i.e., more relevant documents in aggregated retrieval results obtained by multiple queries in a session). Diverse query generation is important because query documents typically contain several different aspects (or topics) and different types of relevant documents may be related to these aspects. To identify aspects, 500 top terms based on their tf-idf rank, are clustered into $ n $ sets with respect to their similarity. Each distinct sets of terms represents one query aspect, then top $ k $ retrieved documents for each sub-query consider as pseudo-relevant documents (PRD) and those ranked below the top $ k $ are non-relevant documents (NRD). Then the query is generated by decision tree. ~\citep{kim2014searching, kim2014diversifying}.
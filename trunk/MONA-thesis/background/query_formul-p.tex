%Query or topic is the the request for information in any retrieval system. An effective query can lead in retrieving the required information. Queries formulated by users are not usually optimal for retrieval process, therefore, they need to be transformed by techniques such as query expansion or query reduction. 
The patent prior art search begins with a full patent application as a query. A full text as a query is a challenge compared to a classical IR because it is not focused on the information that a user needs. In order to achieve good retrieval results, extracting the best representative text with the proper weights is important. In this section, we discuss initial attempts to formulate an effective query out of a patent query~as~follows:
%The query created before the retrieval can be modified or enhanced after retrieval. In this section, the initial query formulation will be discussed and in the next section, post-retrieval query reformulation will be covered. 
\\\\
\textbf{The Best Field to Extract Query Terms}
\ \\
Patents are structured documents and they consist of several different sections: title, abstract, description, and claims. Different sections use different type of language for describing the invention. The abstract and description use more technical terminology while the claim usually uses legal jargon. Structured indexing keeps the field structure in the index that allows searching in specific fields instead of searching in a full document. In addition, separate fields for meta-data (Section \ref{sec:metadata} ) like IPC code and author can be used to improve the retrieval effectiveness~\citep{magdy2010exploring}. 

Early patent search tasks mainly considered claims to build the query, the same as what examiners start in the novelty process~\citep{konishi2005query, takaki2004associative, mase2005proposal, fujii2007enhancing}, whereas recent works show that building queries from description field is more useful in patent retrieval (considering background summary in US patents equivalent to description field in European patents)~\citep{xue2009transforming, xue2009automatic, mahdabi2011building}. Another research shows that extracting terms according to their TF-IDF scores from every field of the query patent, and giving higher importance to the terms extracted from the abstract, claims, and description fields than to the terms extracted from the title field, is an effective way of constructing a search query~\citep{cetintas2012effective}. Another experiment shows that discarding description from query improves the MAP up to 30\% because the description contains more noise than information~\citep{gobeill2010simple}. They also show that claims are more informative and the title is poorly informative in retrieval.   
\\\\
\textbf{Query Formulation Using Phrases}
\ \\
Most of query formulation techniques rely on terms; however, formulating queries using phrases has recently obtained encouraging results~\citep{becks2010phrases}. According to early results, an NLP-based grouping of terms can increase the performance compared to the bag-of-words approach~\citep{osborn1997evaluating}. For example, we can improve the retrieval effectiveness by adding syntactic phrases in the form of dependency triples, to a bag-of-words representation~\citep{d2011combining}. Key phrase extraction (KPE) algorithm is another way to form a query based on phrases; a list of phrases --- generated by a KPE algorithm --- can succinctly represent a complex and lengthy patent~\citep{verma2011applying}.
\\\\
\textbf{Diverse Query Generation}
\ \\
\cite{kim2014diversifying} recently worked on generating diverse queries from the patent query that can improve overall retrieval effectiveness in sessions rather than generating a single best query that can retrieve more relevant documents from a single retrieval result (i.e., more relevant documents in aggregated retrieval results obtained by multiple queries in a session). Diverse query generation is important because query documents typically contain several different aspects and different types of relevant documents may be related to these aspects. To identify aspects, 500 top terms based on their TF-IDF rank, are clustered into $ n $ sets with respect to their similarity. Each distinct sets of terms represents one query aspect, then top $ k $ retrieved documents for each sub-query are considered pseudo relevant documents and those ranked below the top $ k $ are considered irrelevant documents. Finally, the query is generated by a decision tree~\citep{kim2014searching, kim2014diversifying}.
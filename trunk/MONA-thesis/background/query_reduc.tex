In general, retrieval effectiveness for long queries is often lower than retrieval effectiveness for shorter keyword queries because the additional information provided in verbose queries is more likely to confuse current search engines rather than help them. Query reduction (QR), a technique for dropping unnecessary query terms from long queries, improves the performance. 

A common approach to reduce verbose queries is selecting a subset of a long query (or sub-query). A search engine performs more precisely when just the key concepts are used as a query rather than a long query. Hence, the identification of the key query concepts has a positive impact on the retrieval performance for verbose queries. Extracting the key query concepts can be done by learning to identify key concepts in long queries using a variety of features~\citep{bendersky2008discovering}. We can choose effective subsets in a query by analysing all the subsets of terms from the original query (sub-queries), and identifying the most promising sub-query to replace the original long query. For ranking sub-queries, an algorithm based on the SVM classification is used~\citep{kumaran2009reducing}. In this approach, the quality of query reduction depends on the performance of the predictor and ranking algorithm~\citep{balasubramanian2010exploring}. 

As the other approach, we can use query term ranking techniques to select effective terms from a verbose query by ranking them. A vast number of rankings are possible given different settings of individual term weights; for example, we can train a regression model to weight all query words of a verbose query~\citep{lease2009regression}. We can also assign weights to concepts by learning the importance of concepts underlying
the verbose query~\citep{bendersky2010learning}.

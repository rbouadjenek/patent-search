The main textual content of patent documents is known to be difficult to process with traditional text processing and text retrieval techniques; however, patents contain additional material, namely, tables, mathematical and chemical formulas, citations, technical drawing, meta-data (e.g., applicant, inventor, IPC codes, and publication date) that can be used to improve the retrieval. We explain the non-text information (e.g., meta-data) in patents that are used to improve the retrieval performance as~follows:
\paragraph{The Use of Citation}
\ \\
The most successful use of meta-data to date is the citation lists in order to learn patterns of relevance~\citep{lupu2013patent}. The patent collection is a very dense network of citations that creates a set of interrelations and can be exploited during a prior art search. The large majority of patents continue the previous works and patents. The citation relations make this development process
visible. Similarly, fundamental patents, which open a new technology, are exceptional; they tend to be cited very frequently in the whole sub-field during years. A citation graph of a patent collection is used for identifying patent thickets (i.e., the patent portfolios of several companies overlap on a similar technical aspect). Related patents can be inferred from the overall citation network of a patent collection. If a new patent applicant belonging to this patent ticket appears, it is very likely that the most relevant prior art documents are already
present in this patent thicket~\citep{lopez2009multiple}.

The patents cited in the description of the topic patent are used as relevant documents, because citations are usually prior arts for a citing patent. Only citations which are in the collection can be helpful in the retrieval process. The idea of PageRank --- identifying authoritative pages by analysing hyperlink structure on World Wide Web --- can be used for citations. A patent, which is cited by a large number of other patents, is more important. Text-based and citation-based scores are combined to compute the ranking score for the documents~\citep{fujii2007enhancing, fujii2007integrating}.

Citation texts for patents are a whole paragraph. Therefore, for each patent document presented and cited in the collection, the entire paragraph of citation can be appended to the textual material of the cited patent. A boolean feature uses to indicate weather a cited patent in query patent has retrieved, then this document can get a higher weight at any future post-ranking process. Due to the limited number of citation texts, this approach showed just a trivial improvement~\citep{lopez2009multiple}. However, citation information is not always presented in the patent application and this method cannot be used in real-life patent search and initial citations by the applicants may not consider relevant by patent examiners~\citep{magdy2010applying, magdy2011simple}. 
Similar tasks by \cite{gobeill2010simple} and \cite{gurulingappa2010prior} also indicate improvement in MAP and recall using citations in patent~retrieval. 
\paragraph{The Use of IPC Codes}
\ \\
Patents are classified by the patent offices into large hierarchical classification schemes based on their area of technology. The use of patent
classification has two major benefits. The first is that the classifications provide access to concepts rather than words, such that even
if the same word or phrase is commonly used in two technology areas, patent classifications will provide the context of its use. In effect, they
allow the search space of patents to be reduced, by allowing the user to exclude from the search process patents in classes not related to
the search topic at hand~\citep{lopez2010patatras}. The second major benefit is the language independence provided by classifications, as classification symbols can be mapped to multiple languages~\citep{DBLP:conf/clef/DhondtV10}. This allows patent searchers to conduct reasonably effective retrieval even in languages that they do not understand. All previous work, considered IPC code in their search, reported improvement in retrieval effectiveness~\citep{harris2010comparison, harris2011using, harris2009role, fujita2005revisiting, graf2010knowledge, herbert2010prior, kang2007cluster, verma2011applying}. It has also reported that using complete IPC code leads in better results than just 4-digit code~\citep{ gobeill2010simple}.
\paragraph{The Use of Images}
\ \\
For the purposes of the search for innovation, we are interested in all forms of information. Some technology areas rely information present in images (flowcharts and diagrams), so, beyond text data, image processing tasks also can contribute to the search~\citep{lupu2013patent}. A graph-based measure has a higher discriminative power, but higher computational costs than the text-based measures~\citep{lupu2013evaluating}.

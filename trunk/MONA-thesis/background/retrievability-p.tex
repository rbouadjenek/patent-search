Rerievability is specifically critical in recall oriented applications like patent retrieval or legal settings. In these cases, the focus of a system is to retrieve all documents that are relevant rather than to retrieve a subset of documents that the best satisfy the query intent.
Hence, all documents should at least potentially be retrievable via correct query terms. Designing retrieval systems for recall oriented tasks has been emphasised in recent years~\citep{fujii2007introduction, kontostathis2008effect}. Before designing a new or using an existing retrieval system for recall oriented applications, one needs to analyse the effects of the retrieval system bias as well as the overall retrievability of all documents in the collection using the retrieval function at hand.

In retrievability, we analyse documents specifically with respect to relevant and irrelevant queries to identify whether highly retrievable documents are really highly retrievable, or whether they are simply more accessible from many irrelevant queries rather than from relevant queries. Experiments show that about 90\% of patent documents which are highly retrievable across all types of queries, are not highly retrievable on their relevant query sets~\citep{bashir2009analyzing}.

Experiments with different collections of patent documents suggest that query expansion with pseudo relevance feedback can be used as an effective approach for increasing the findability of individual documents and decreasing the retrieval bias. Pseudo relevance feedback documents are identified using cluster-based~\citep{bashir2009improving} or terms-proximity-based methods~\citep{bashir2010improving}.

Another study~\citep{bache2010improving} analyses the relationship between retrievability and effectiveness-based measures (Precision, Mean Average Precision). Results show that the two goals of maximising access and maximising performance are quite compatible. They further conclude that a reasonably good retrieval performance can be obtained by selecting parameters that maximise retrievability (i.e., when there is the least inequality between documents according to Gini coefficient given the retrievability values). Their results support the hypothesis that retrieval functions can be effectively tuned using a retrievability-based measure without recourse to relevance judgments, making it an attractive alternative for automatic evaluation.

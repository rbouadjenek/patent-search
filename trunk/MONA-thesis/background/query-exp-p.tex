Although a query is very long in patent prior art search, a significant term mismatch between queries and relevant documents has reported in~\citep{roda2010clef, magdy2010exploring}. The QE is a suggested solution to cope with the term mismatch problem; however, most of the QE techniques were ineffective to improve the performance in patent domain~\citep{kishida2003experiment, konishi2005query}. We review previous works on the QE techniques for the patent search here.
\paragraph{Query Expansion by Pseudo Relevance Feedback}
\ \\
Previous studies~\citep{magdy2011study} have discussed that PRF is ineffective for patent prior art search. Since the retrieval effectiveness is low at initial retrieval, the assumption that top $ k $ documents are relevant is invalid and leads in adding noise to the query; hence, the improvement using PRF is insignificant. The solutions proposed to cope with this problem are as follows:
\begin{list}{-}{}
\item \textbf{Selecting documents for PRF based on cluster analysis:} In this approach, a document that can cluster lots of high similar documents is considered relevant and a document that has no nearest neighbour or some neighbours with low similarity is considered irrelevant~\citep{lee2008cluster}. In patent domain, where there is a large vocabulary diversity for expressing an invention, the idea can be improved by intra-cluster similarity rather than only on the basis of their size~\citep{bashir2009improving}. 
\item \textbf{Selecting patents for PRF based on their similarity with query patent via specific terms:} In this approach, patents for PRF are identified based on their similarity with query patents over a subset of terms, rather than the overall document similarity. The succession of this approach highly depends on selecting appropriate terms from patent query, which produce the best PRF candidates that can help in improving retrievability during QE~\citep{bashir2010improving}. Experiments show a significant improvement for Gini coefficient, which is used to measure retrievability, but there is no report on other measures.
\item \textbf{Identyfying expansion terms: } 
\ \\
Term proximity information can be used to identify expansion terms. Given a patent query, first, an initial query is generated by taking, for example, claim terms; then a query-specific lexicon based on definitions of the IPC is created.
Among many terms in the lexicon, only expansion terms identified by two adjacency operators used in patent examination\footnote{Patent examiners use term proximity heuristics in their
searches in Boolean retrieval model in order to reward a
document where the matched query terms occur close to
each other. Two forms of adjacency operators are used in
Boolean retrieval model to address proximity. `ADJn' operator which searches for terms within \textit{n} words proximity in
the order specified, and `NEARn' operator, which searches
for the terms within \textit{n} words, in either order.} (i.e., `ADJn' and `NEARn') are used for query expansion~\citep{mahdabi2013leveraging}.
\item \textbf{Predicting the effectiveness of feedback documents: } 
\ \\
In patent retrieval, the MAP is very low at initial retrieval; hence the top retrieved documents are not essentially relevant. 
As a result, there is a high
chance that we use irrelevant documents for expansion in PRF. Recently,
machine learning methods like regression are used to improve the PRF by predicting the effectiveness of a feedback document~\citep{mahdabi2012learning}.
\end{list}
Random indexing to identify terms to use for query expansion~\citep{sahlgren2002english} and expansion using noun phrases~\citep{mahdabi2012automatic} are the other techniques to improve the effectiveness of standard query expansion for prior art search. 
\paragraph{Query Expansion by External Resources}
\ \\
Some external resources like WordNet~\citep{miller1990introduction}, which were reported to improve retrieval effectiveness in several IR research investigations, show insignificant change to overall retrieval effectiveness, but a degree of improvement for some topics in patent domain. \cite{magdy2011study} applied the idea of automatically generating the synonyms set (SynSet) using parallel manual translations to create possible synonyms sets (in CLEF-IP collection, some of the sections in some patents are translated into three languages: English, French, and German). Although this idea presents better results compared to WordNet, there is still a little improvement in retrieval effectiveness. The only QE task, which achieves the best results, uses a combination of PRF and QE with translation of terms and phrases from German and French~\citep{jochim2011expanding}.

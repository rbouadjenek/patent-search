\chapter*{Acknowledgments}
\addcontentsline{toc}{chapter}{Acknowledgments}

Upon the accomplishment of this thesis, I would like to thank people who influenced and enlightened my academic path. 
First, I owe a debt of gratitude to the main director of this research: Scott Sanner. I have been quite fortunate to work on another research advised by him before he leaves NICTA/ANU to Oregon State University. He highly encouraged me  (1) to work hard, but with enthusiasm, (2) to feel responsibility for honest contribution to science, (3) to be an independent and efficient thinker, (4) to deal with a research problem by proposing to-the-point research questions, (5) to be brave enough to examine new ideas and to seek for the best solution, and (6) to collaborate with other researchers and to be generous in sharing the success of my work with others. I hope I have learned enough to continue my academic career even without his unselfish support. I believe that the science world needs more people like Scott. 

I thank my other committee members, Tom Gedeon (chair of my panel), Gabriela Ferraro, and Hanna Suominen for their supports over the course of my MPhil. I had always Tom's support though he has been too busy. Gabriela has been always for me if she could and Hanna kindly provided a thorough review of my thesis. 
%I appreciate Hanna proposing to review by thesis because her comments improved my thesis. 
I am indebted to them all for their time and effort they spent. 

I acknowledge the academic and technical support provided by the
National ICT of Australia (NICTA) and the Australian National University (ANU)
and thank them for their support in my research. Specially, I would like express my gratitude to Bob Williams --- the leader of the Machine Learning (ML) group of NICTA. He always encouraged students, including me, to participate in NICTA events (e.g., NICTA ML Retreat and Review); he also supported me to attend NICTA ML Summer School. He created such a positive research atmosphere and culture with a collection of excellent researchers that made my tenure at NICTA an extraordinary fruitful experience. I learned a lot discussing with smart researchers on my neighbourhood like Scott Sanner, Justine Domke, and Aditya Menon and attending their seminars, talks, and reading groups. 

During my one-year MPhil program, I met many successful IR researchers like Paul Thomas (and his great IR \& friends seminars), Milad Shokuhi, and Leif Azzopardi who left a strong impact on my career. I would also thank David Hawking for his smart comments on my work. Dave and Scott's support encouraged me to have a submission to SIGIR that got accepted $\ddot\smile$. 

I would like also mention about anonymous reviewers of my SIGIR paper  for their useful comments. Their words warm my heart to stay in research and have more contribution to Computer Science/IR community. I always feel confident to go forward when I read their following comment on my paper: "Although the amount of work on patent query reformulation was large since 2010, but none of this work did that simple but interesting study. The study and results are very interesting and can be very useful for patent examiners in practical search situations". I owe the presentation of my work at Chile (Santiago) to both generous SIGIR travel grant and ANU fund. 

I appreciate Reda Bouadjenek, for his contribution to the baseline IR framework. I would like also thank Ehsan Abbasnejad for patiently answering my questions (Of course, Scott taught me a priceless lesson, which will be with me forever: "Google is the best~friend, any time and anywhere!" $\ddot\smile$).

And finally, I am grateful to all of my family and friends for being there for me whenever I needed. An special thanks to my dearest Ali for being always the main support after my parents and for respecting my attempts to follow my goals and values. I owe a debt of love to my parents who has been the main motivation behind every single step in my academic life by their famous recommendation: "Go forward for the highest possible academic degree and never stop learning or experiencing a new path". 
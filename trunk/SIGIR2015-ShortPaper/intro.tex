\begin{comment}
Patents are used by legal entities to legally protect their
inventions and represent a multi-billion dollar industry of licensing
and litigation. In 2013, 302,948 patent applications were approved
in the US alone%
\footnote{http://www.uspto.gov/web/offices/ac/ido/oeip/taf/ us\_stat.htm%
}, a number that has doubled in the past 15 years. Given that a single
existing patent may invalidate a new patent application, helping inventors
assess the patentability of an idea through a patent prior art search
before writing a complete patent application is an important task.
\end{comment}

Patent prior art search involves finding previously granted patents,
or any published work, such as scientific articles or product
descriptions that may be relevant to a new patent application. The
objective and challenges of standard formulations of patent prior art
search are different from those of standard text and web search
since~\cite{magdy2012toward}: (i) queries are reference patent
applications, which consist of documents with hundreds or thousands of
words organized into several sections, while typical queries in text
and web search constitute only a few words; and (ii) patent prior art
search is a recall-oriented task, where the primary focus is to
retrieve all relevant documents at early ranks, in contrast to text
and web search that are precision-oriented, where the primary goal is
to retrieve a subset of documents that best satisfy the query
intent. Another important characteristic of patent prior art search is
that, in contrast to scientific and technical writers, patent writers
tend to generalize and maximize the scope of what is protected by a
patent and potentially discourage further innovation by third parties,
which further complicates the task of formulating queries.

\begin{comment}
A patent is a set of exclusive rights granted to an inventor to
protect their invention for a limited period of time. An important
requirement for a patent to be granted is that the invention, it
describes, is novel which means there is no earlier patent,
publication or public communication of a similar idea. To ensure the
novelty of an invention, patent offices as well as other Intellectual
Property (IP) service providers mainly perform a search called `prior
art search'. The purpose of `prior art search' is finding all relevant
patents which may put the patent application at the risk of novelty
invalidation or at least have common parts with patent application and
should be cited~\cite{magdy2012toward}~\cite{piroi2010clef}.

Patent retrieval has three main characteristics which makes it
difficult compared to other IR applications: (1) the search starts
with a query as long as a full patent application that helps users
--usually patent examiners, inventors, or lawyers-- avoid spending
long hours to formulate a query; (2) it is recall-oriented, where not
missing relevant documents is more important than appearing relevant
documents at top of the list; (3) unlike the web application in which
authors tend to highlight their work to be easily found through search
engines, authors of the patents prefer to use a vague language to
avoid the invalidation of their idea.
\end{comment}

%Many works has been conducted to improve the patent retrieval effectiveness so far. However, either the results showed quite small improvement or the proposed methods were complicated and computationally expensive. 
%
% I don't understand the difference between term selection and query 
% reformulation.  I think they're more or less synonymous.  -Scott
%
%Existing work on patent search largely falls into four main
%categories~\cite{lupu2013patent}: query reformulation (query expansion
%and query reduction), query suggestions, using patent meta-data and
%images for retrieval~\cite{lupu2013evaluating}, and cross-language
%approaches~\cite{magdy2014studying}.

%Applying standard information retrieval (IR) techniques to patent search is not effective and needs applying supplementary methods to improve the effectiveness. Although lots of methods have been proposed in recent years, reported results for different tasks of patent search show lower retrieval effectiveness compared to other IR applications~\cite{lupu2013patent}.  

% Let's focus directly on query reformulation --- other discussion is
% somewhat orthogonal to the very specific task of this paper.
%
% We need some specific citations for for query reformulation and
% a clear transition that explains why our work is different from
% existing investigations -- I currently have a weak transition,
% but we could probably do better.  -Scott
%
In this work, we focus on the task of query
reformulation
%~\cite{Baeza-Yates2011} 
specifically applied to patent
prior art
search~\cite{mahdabi2014patent,xue2009transforming}.  While
prior work has largely focused on specific techniques for query
reformulation, in Section \ref{Sec:OracularTermSelection}, we first
build an oracular query formed from known relevance judgments for the
CLEP-IP 2010 Prior Art test collection~\cite{piroi2010clef} in an attempt
to derive an upper bound on performance of standard Okapi BM25 and
Language Models (LM) retrieval algorithms for this task.  Since the
results of this evaluation suggest that query reduction methods can
outperform state-of-the-art prior art search performance, in Section
\ref{sec:AutomatedReduction} we proceed to analyze four simple
automated methods for identifying terms to remove from the original
patent query.  Finding that none of these methods seems to
independently yield promise for query reduction that strongly
outperforms the baseline, in Section
\ref{sec:SemiAutomatedInteractiveReduction} we evaluate an alternative
interactive feedback approach where terms are selected from only the
first retrieved relevant document.  Observing that such simple
interactive methods for query reduction with a standard LM retrieval
model outperform highly engineered patent-specific search systems from
CLEF-IP 2010, we conclude that interactive methods offer a promising
avenue for simple but highly effective term selection in patent prior
art search.

%However, before giving more technical details, in the next section, we first introduce the IR system and the settings we use throughout this paper.

%This paper is organized as follows: In Section \ref{Sec:BaselineIRFramework}, we introduce the IR system and the settings we use in this paper; in Section \ref{Sec:OracularTermSelection} we introduce two oracular methods used to get an upper bound on the performance; in Section \ref{Sec:QueryReduction} we introduce query reduction methods; in Section \ref{Sec:RelatedWork} we discuss the related work; and in Section \ref{Sec:Conclusion} we conclude the paper.

\begin{comment}
In this work, we mainly textitasized on the problem from the term analysis perspective which ended in an effective minimal relevance feedback method. We investigated the influence of term selection on retrieval performance on the CLEF-IP Prior Art test collection,  using the Description section of the patent query with Language Model (LM) and BM25 scoring functions. We found that an oracular relevance feedback system which extracts terms from the judged relevant documents far outperforms the baseline and  performs twice as well on MAP as the best competitor in CLEF-IP 2010.  We find a very clear term selection value threshold for use when choosing terms.  A much more realistic approach in which feedback terms are extracted only from the first relevant document retrieved, still outperforms the winner.   We noticed that most of the useful feedback terms are actually present in the original query and hypothesized that the baseline system could be substantially improved by removing negative query terms.  We tried four different approaches to identifying negative terms but were unable to improve on the baseline performance with any of them.
\end{comment}

\begin{comment}
- Prior work suggests query reformulation is critical for patent prior art search.  Different works suggest query expansion, query reduction, or both.

- Using an ideal query formed from the relevance judgments and query patent, we see that selecting a subset of terms within the query patent are sufficient to achieve high MAP --- 1.5x better than the best CLEF-IP 2010 results and 3x better than a BM25 or LM baseline using all query patent terms.  This suggests query reduction should suffice for effective prior art patent retrieval.

- Further analysis shows an unexpected steep dropoff in performance when the ideal query is "polluted" with additional terms from the original patent query suggesting that very precise methods for eliminating poor query terms are needed in the reduction process.

- Standard proposals for automated query reduction yield no benefit over the baseline query.  Anecdotal analysis of a few patents further reinforces the difficulty of automatically identifying terms to eliminate.

- However, using a simple interactive relevance feedback model using only the first relevant document achieves a MAP score better than any of the previous CLEF-IP 2010 competitors.  Since baseline methods return a relevant patent ~80% of the time in the first 10 results and 90% of the time in the first 20 results, such an interactive approach requires relatively low user effort while achieving state-of-the-art performance.

- Overall the difficulty of automatic query reduction and the surprising effectiveness of a minimal relevance feedback method suggests the potential importance of interactive retrieval methods for patent prior art search.
\end{comment}


Query reduction is a solution for problems with using a full verbose patent as a query: it is not focused on information needed by the user, and a verbose query may cover more than one topic.
\paragraph{Query Segmentation:} 
\ \\
Decomposing each patent query into coherent sub-topics segments --- using TextTiling~\citep{hearst1997texttiling} -- is a solution to make long ambiguous queries focused on the information need. Sub-topic segments can be used as separate queries (query stream) for initial retrieval, then the retrieval results from each of the individual streams are merged to construct the final ranked list for the whole original query. Using each sub-topic as a query stream enables a retrieval model to retrieve related documents from the collection in a more precise way and also allow the PRF algorithm to work on a more focused set of pseudo-relevant documents~\citep{takaki2004associative, ganguly2011united}. Another work adapted pseudo relevance feedback for query reduction by decomposing a patent application into constituent text segments and computing the LM similarities by calculating the probability of generating each segment from the top ranked documents. The least similar segments from the query removed from the query, hypothesizing that removal of segments most dissimilar to the pseudo-relevant documents can increase the precision of retrieval by removing non-useful context, while still retaining the useful context to achieve high recall as well~\citep{ganguly2011patent}.
\paragraph{Patent Summarisation:} 
\ \\
This approach assumes that the patent summary (using TextTiling) reflects the main topic as well as the subtopics of a patent document in a concise manner. Then, language model for the query, collection, and each summary are generated~\citep{mahdabi2011report}. 
\paragraph{Query Term Selection:}
\ \\
Identifying useful query terms and giving them higher weights is important to build an effective query. The simplest proposed approach is weighting terms in the query based on their perceived significance in the target corpus, combined with their significance in the query~\citep{itoh2003term}. The problem with this method is that it does not take into account the fact that some terms, while being important to the definition of the request for information, may not necessarily appear in the target set at all. For query term selection purposes, it would seem more useful to weight them based only on the genre to which the query belongs, rather than the genre of the target collection. The enhanced version of selecting the most discriminative terms for each topic patent is to compute Kullback-Leibler divergence (KLD)~\citep{kullback1951information} between the language model of the query and the whole collection as follows:
%\[ 
\begin{equation}
\label{eq:kld}
 KLD(P_{Q}(t)||P_{C}(t)) = P_{Q}(t)\log\Big(\frac{P_{Q}(t)}{P_{C}(t)}\Big),  
\end{equation}
% \tag{2-1}\label{eq:kld}
% \]
where $ P_{Q} $ is the probability of each term $ t $ within the patent topic $ q $, and $ P_{C} $ is the probability of the same term $ t $ within the whole collection. 
By applying the Equation~\ref{eq:kld}, it is possible to rank all the terms from the patent topic according to their importance within the query. After ranking the terms by their divergence, only terms with divergence above an specific threshold are selected. Thus, we can build queries that contain the most discriminative terms in different fields of query, which appear frequently in the query, but not so frequently in the collection. So, it helps to retrieve the most relevant patents to a given topic~\citep{perez2010using}. It is possible to exploit the knowledge of IPC meta-data into the query model~\citep{mahdabi2011building} as follows:
%\[ 
\begin{equation}
\label{eq:IPCmodel}
 P_{Q}(t) = \lambda\frac{c(t,Q)}{|Q|}+\frac{(1-\lambda)}{N}\sum_{D\in IPC_{Q}}\frac{c(t,D)}{|D|} , 
\end{equation}
 %\tag{2-2}\label{eq:IPCmodel}
% \]
where $ c(t,Q) $ is the term frequency of the term $ t $ in the query patent document, $ |Q| $ is the length of the query patent, $ N $ is the size of the relevant cluster with the same IPC code as the query, and $ \lambda $ is the smoothing parameter.
%\item Used patent structure, linguistic clues, and word relations to identify important terms. Showed that keyword dependency relation
%approach achieved 13-18\% improvement in MAP over the traditional tf-idf based term weighting method when a single field is used for query formulation. Obtained 42-46\% improvement in MAP when used additional terms through pattern-based semantic tagging.~\citep{nguyen2012query}


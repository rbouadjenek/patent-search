The interest in multilingual patent search arises from their international and multilingual nature (the European patent office-EPO-makes patent text available in three languages: English, French, and German). Patents on the same topic may be published in different countries in different languages, and it is important for patent examiners to be able to locate relevant existing patents whatever language they are published in. Therefore an important topic in patent retrieval is Cross-Language Information Retrieval (CLIR), where the topic is a patent application in one language and the objective is to find relevant prior-art patents in another language ~\citep{lupu2013patent,joho2010survey, roda2010clef, DBLP:conf/clef/PiroiLHSMF12}. In recent years machine
translation (MT) has become established as the dominant technique for translation in CLIR, which usually achieve better CLIR effectiveness than dictionary-based translation (DBT) methods. However, translation using MT is time consuming and resource intensive for cross language patent retrieval (CLPR), where the query text can often take the form of a full patent application running to tens of pages. Applying IR text pre-processing like stop word removal and stemming to the MT training corpus prior to the training phase can lead to a significant decrease in the MT computational~\citep{magdy2013studying}.



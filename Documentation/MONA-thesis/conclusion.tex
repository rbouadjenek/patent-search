\chapter{Conclusions}
\label{cha:conc}
%\textcolor{red}{Mona: I will compose this chapter after we confirmed about all chapters!}\\
%Summary your thesis and discuss what you are going to do in the future in Section~\ref{sec:future}.

\section{Overview}
\label{sec:overview}
In this thesis, we investigated the reasons that patent prior art searches 
are less effective than the other web searches on CLEF-IP 2010 data collection. 
We started with recognizing errors due to data curation and baseline settings that 
make small portions of the whole retrieval errors. The main portion of the errors are 
due to term matching process in retrieval ranking functions. 
Hence, we looked at the patent prior art search from
a term selection perspective. While previous works proposed
different solutions to improve retrieval effectiveness, we 
focused on term analysis of the patent query and top-100 retrieved patents. 
After defining an oracular query based on
relevance judgements, we established both the sufficiency
of the standard LM retrieval scoring models and query reduction 
methods to achieve state-of-the-art patent prior art
search performance. After finding that automated methods 
for query reduction approaches fail to offer significant
performance improvements, we showed that we can double
the MAP with minimum user interaction by approximating
the oracular query through a relevance feedback approach
with a single relevant document. Given that such simple 
interactive methods for query reduction with a standard LM
retrieval model outperform highly engineered patent-specific
search systems from CLEF-IP 2010, we concluded that interactive 
methods offer a promising avenue for simple but
highly effective term selection in patent prior art search.
 
%\section{Summary}
%\label{sec:summary}

\section{Contributions}
\label{sec:contributions}

\section{Future Work}
\label{sec:future}





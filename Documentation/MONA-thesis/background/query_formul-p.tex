%Query or topic is the the request for information in any retrieval system. An effective query can lead in retrieving the required information. Queries formulated by users are not usually optimal for retrieval process, therefore, they need to be transformed by techniques such as query expansion or query reduction. 
The patent prior-art search scenario begins with the full patent application as a query. A full text as a query is a challenge compared to a classical IR, since it is not focused on the information that the user needs. In order to achieve good retrieval results, it is important to extract the best representative text with the proper weights. Therefore, query generation based on query document is essential to reduce the difficulty of formulating effective queries by users. 
%The query created before the retrieval can be modified or enhanced after retrieval. In this section, the initial query formulation will be discussed and in the next section, post-retrieval query reformulation will be covered. 
\paragraph{Terms Selection}
\ \\
Identifying useful query terms and giving them higher weights is important to build an effective query. The simplest proposed approach is weighting terms in the query based on their perceived significance in the target corpus, combined with their significance in the query~\citep{itoh2003term}. The problem with this method is that it does not take into account the fact that some terms, while being important to the definition of the request for information, may not necessarily appear in the target set at all. For query term selection purposes, it would seem more useful to weight them based only on the genre to which the query belongs, rather than the genre of the target collection. The enhanced version of selecting the most discriminative terms for each topic patent is to compute Kullback-Leibler divergence (KLD)~\citep{kullback1951information} between the language model of the query and the whole collection as follows:
%\[ 
\begin{equation}
\label{eq:kld}
 KLD(P_{Q}(t)||P_{C}(t)) = P_{Q}(t)\log\Big(\frac{P_{Q}(t)}{P_{C}(t)}\Big),  
\end{equation}
% \tag{2-1}\label{eq:kld}
% \]
where $ P_{Q} $ is the probability of each term $ t $ within the patent topic $ q $, and $ P_{C} $ is the probability of the same term $ t $ within the whole collection. 
By applying the Equation~\ref{eq:kld}, it is possible to rank all the terms from the patent topic according to their importance within the query. After ranking the terms by their divergence, only terms with divergence above an specific threshold are selected. Thus, we can build queries that contain the most discriminative terms in different fields of query, which appear frequently in the query, but not so frequently in the collection. So, it helps to retrieve the most relevant patents to a given topic~\citep{perez2010using}. It is possible to exploit the knowledge of IPC meta-data into the query model~\citep{mahdabi2011building} as follows:
%\[ 
\begin{equation}
\label{eq:IPCmodel}
 P_{Q}(t) = \lambda\frac{c(t,Q)}{|Q|}+\frac{(1-\lambda)}{N}\sum_{D\in IPC_{Q}}\frac{c(t,D)}{|D|} , 
\end{equation}
 %\tag{2-2}\label{eq:IPCmodel}
% \]
where $ c(t,Q) $ is the term frequency of the term $ t $ in the query patent document, $ |Q| $ is the length of the query patent, $ N $ is the size of the relevant cluster with the same IPC code as the query, and $ \lambda $ is the smoothing parameter.
%\item Used patent structure, linguistic clues, and word relations to identify important terms. Showed that keyword dependency relation
%approach achieved 13-18\% improvement in MAP over the traditional tf-idf based term weighting method when a single field is used for query formulation. Obtained 42-46\% improvement in MAP when used additional terms through pattern-based semantic tagging.~\citep{nguyen2012query}


\paragraph{Using Phrases instead of Terms}
\ \\
Most of query formulation techniques rely on terms, but encouraging results have been obtained using phrases recently~\citep{becks2010phrases}. Early results demonstrated that an NLP-based grouping of terms can increase the performance compared to the bag-of-words approach, though the increase is smaller than in a non-patent collection~\citep{osborn1997evaluating}. Another task could improve retrieval effectiveness by adding syntactic phrases in the form of dependency triples, to a bag-of-words representation~\citep{d2011combining}. Key Phrase Extraction (KPE) algorithms is another way to form a query based on phrases. A list of phrases, generated by a KPE algorithm, can succinctly represent a complex and lengthy patent. ~\citep{verma2011applying}.

\paragraph{Diverse Query Generation}
\ \\
In this approach, the focus is on generating diverse queries that can improve overall retrieval effectiveness in sessions rather than generating a single best query that can retrieve more relevant documents from a single retrieval result (i.e., more relevant documents in aggregated retrieval results obtained by multiple queries in a session). Diverse query generation is important because query documents typically contain several different aspects (or topics) and different types of relevant documents may be related to these aspects. To identify aspects, 500 top terms based on their tf-idf rank, are clustered into $ n $ sets with respect to their similarity. Each distinct sets of terms represents one query aspect, then top $ k $ retrieved documents for each sub-query consider as pseudo-relevant documents (PRD) and those ranked below the top $ k $ are non-relevant documents (NRD). Then the query is generated by decision tree. ~\citep{kim2014searching, kim2014diversifying}.
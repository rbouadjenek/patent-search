Rerievability is specifically critical in recall oriented application, such as patent retrieval, or legal settings. In these cases, the focus of a system is not so much on providing the best document to answer a specific information
need (as e.g. in Web search settings), but to retrieve all documents that are relevant. Thus, all documents should at least potentially be retrievable via correct query terms. Designing retrieval systems for recall oriented tasks has been emphasized in recent years~\citep{fujii2007introduction, kontostathis2008effect}, but before designing a new or using an existing retrieval system for recall oriented applications one needs to analyse the effects of the retrieval system bias as well as the overall retrievability of all documents in the collection using the retrieval function at hand.

Analysing retrievability of documents specifically with respect to relevant and irrelevant queries to identify whether highly retrievable documents are really highly retrievable, or whether they are simply more accessible from many irrelevant queries rather than from relevant queries, revealed that 90\% of patent documents which are highly retrievable across all types of queries, are not highly retrievable on their relevant query sets~\citep{bashir2009analyzing}.

Experiments with different collections of patent documents suggest that query expansion with pseudo-relevance feedback can be used as an effective approach for increasing the findability of individual documents and decreasing the retrieval bias. Pseudo-relevance feedback documents were identified using cluster-based ~\citep{bashir2009improving} or terms-proximity-based methods ~\citep{bashir2010improving}.

Another study analysed the relationship between retrievability and effectiveness-based measures (Precision, Mean Average Precision)~\citep{bache2010improving}. Results showed that the two goals of maximizing access and maximizing performance are quite compatible. They further concluded that reasonably good retrieval performance can be obtained by selecting parameters that maximize retrievability (i.e., when there is the least inequality between documents according to Gini-Coefficient given the retrievability values). Their results support the hypothesis that retrieval functions can be effectively tuned using retrievability-based measure without recourse to relevance judgments, making it an attractive alternative for automatic evaluation.

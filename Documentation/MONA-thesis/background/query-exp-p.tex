In patent domain, query is very long and there is a significant mismatch between queries and relevant documents~\citep{roda2010clef, magdy2010exploring}. Most of 'QE' techniques often did not demonstrate any significant improvement in effectiveness for patent search~\citep{kishida2003experiment, konishi2005query}. Therefore, for an effective query expansion in patent domain, as it will be discussed in this section, specific techniques have been exploited.
\paragraph{Query Expansion by Pseudo Relevance Feedback (PRF)}
\ \\
PRF, in patent domain, is not as effective as in other applications because of the poor effectiveness in initial retrieval. So, the assumption that top $ k $ documents are relevant is wrong and we might add noise in the query, so, the improvement is insignificant. The solutions proposed to cope with this problem are as follows:
\begin{list}{-}{}
\item \textbf{Selecting documents for PRF based on cluster analysis:} a document that can cluster lots of high similar documents considers relevant and a document that has no nearest neighbour or some neighbours with low similarity is irrelevant~\citep{lee2008cluster}. In patent domain, where there is a large vocabulary diversity for expressing an invention, the idea can be improved by intra-cluster similarity rather than only on the basis of their size~\citep{bashir2009improving}. 
\item \textbf{Selecting patents for PRF based on their similarity with query patent via specific terms:} In this approach patents for PRF are identified based on their similarity with query patents over a subset of terms, rather than the overall document similarity. The succession of this approach highly depends on selecting appropriate terms from query patent, which produce the best PRF candidates that can help in improving retrievability during 'QE'~\citep{bashir2010improving}. This set of experiments showed significant improvement for Gini coefficient, which is used to measure retrievability, but there is no report on patent retrieval effectiveness measures.
\item \textbf{Identyfying expansion terms: } 
\ \\
Term proximity information can be used to identify expansion terms. Given a query patent, first an initial query is generated by taking ,for example, claim terms, then a query-specific lexicon that includes the terms from the same IPC patents is  built. Among many terms in the lexicon, only expansion terms identified by two adjacency operators used in patent examination (i.e., “ADJn” and “NEARn”)~\citep{mahdabi2013leveraging}.
\item \textbf{Predicting the effectiveness of feedback documents: } 
\ \\
Regression can be used to predict the effectiveness of a feedback document. Different features can also be used to capture the effectiveness of a feedback document in terms of its performance in query expansion~\citep{mahdabi2012learning}.
\end{list}
Random indexing to identify terms to use for query expansion~\citep{sahlgren2002english}, and expansion using noun phrases ~\citep{mahdabi2012automatic} are the other works to improve the effectiveness of standard query expansion for prior-art search. 
\paragraph{Query Expansion by External Resources}
\ \\
Some external resources like WordNet~\citep{miller1990introduction}, which were reported to improve retrieval effectiveness in several IR research investigations, showed insignificant change to overall retrieval effectiveness, but a degree of improvement for some topics in patent domain~\citep{magdy2011study}. They also applied the idea of automatically generating the synonyms set (SynSet) using parallel manual translations to create possible synonyms sets (In CLEF-IP patent collection, some of the sections in some patents are translated into three languages: English, French, and German). Although this idea presented better results than WordNet, there was no considerable improvement in retrieval effectiveness. The only QE task that achieved the best results, used a combination of PRF and QE with translation of terms and phrases from German and French~\citep{jochim2011expanding}.
